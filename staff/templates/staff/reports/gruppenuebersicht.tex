


% Überarbeitet von Tobias Otterbein 16.09.2013
% Vorlage zur Erstellung der Gruppenübersicht für den Helpdesk

\documentclass[a4paper]{article}

\usepackage[utf8]{inputenc}
\usepackage[T1]{fontenc}
\usepackage[ngerman]{babel}
\usepackage[left=1cm, right=1cm, top=2cm, bottom=2cm]{geometry}
\usepackage{graphicx}
\usepackage{longtable}


\pagestyle{empty}

%% Gibt eine Gruppenzeile aus
% 1: Gruppenname
% 2: Gruppenbild
% 3: Tutorenliste (\Tutor-Tags)
% 4: Raumliste (\Raum-tags
\newcommand{\Gruppe}[4]{
	\includegraphics[width=2cm, height=2cm]{#2}
	 & \parbox{3cm}{\vspace{-1.5cm}\Large \sf{#1}}
	 & \parbox{3cm}{\vspace{-1.5cm}\sf{#4}}
	 & \parbox{7cm}{\vspace{-1.5cm}\sf{#3}}\\
	\hline
}

%% Gibt eine Tutorenzeile aus
% 1: Name
% 2: Handynummer
\newcommand{\Tutor}[2]{
	\noindent{}#1: #2\par
}

\newcommand{\Raum}[1]{
	#1 \par
}

\begin{document}

\begin{longtable}{l l|l|l}

	\sf Bild &  \sf Name &  \sf Räume &  \sf Tutoren\\
	\hline
	\hline

%\Gruppe{Wario}{daten/bilder/supermario/icon_wario}{\Tutor{Peter}{128}\Tutor{Tobias}{5459}}{\Raum{S1|03 105}\Raum{S1|03 105}\Raum{S1|03 105}\Raum{S1|03 105}\Raum{S1|03 105}}

\Gruppe{ {{ group.name|tex_escape }} }{{ group.get_picture_name }}{ \Tutor{ {{ tutor.get_name|tex_escape }} }{ {{ tutor.phone }}} }{ \Raum{ {{ room|tex_escape }} }  }

\end{longtable}
\end{document}

